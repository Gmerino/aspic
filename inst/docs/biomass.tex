\documentclass[fleqn]{article}
\usepackage{color}
\usepackage[a4paper,height=22cm]{geometry}
\usepackage{parskip}
\newcommand{\MSY}{\mathrm{MSY}}
\newcommand{\msy}{_\mathrm{MSY}}
\newcommand{\Rate}{\mathrm{Rate}}
\begin{document}

\title{Biomass dynamic models}
\author{Arni Magnusson}
\maketitle
\tableofcontents

\section{Overview}

Kingsland (1982) gives a review of biomass models.

\newpage

\section{Schaefer (1954)}

\subsection{$\mathrm{Same}+\mathrm{Growth}-\mathrm{Removals}$}

The Schaefer model is:

\begin{eqnarray*}
  B_t &=& B_{t-1} \;+\; rB_t\!\left(\;\!\!1\!-\!\frac{B_t}{K}\right) \;-\; C_t
\end{eqnarray*}

\section{Surplus production}

The second term describes the relationship between surplus production (growth)
and abundance:

\begin{eqnarray*}
  g(B) &=& rB\left(1-\frac{B}{K}\right)\\[1ex]
\end{eqnarray*}

In biological terms, the surplus production combines recruitment, body growth,
and natural mortalities.

Conceptually, the model describes the simplest case of linear recruitment, where
all individuals have the same body weight, and the natural mortality rate is
constant. Individiuals reproduce as 1-year-olds and then die. After a period of
removals, the population rebuilds towards $K$ until the number of individuals
dying from natural causes equals the recruitment.

\subsection{Original equation}

Schaefer (1954, Eq.~2):

\begin{eqnarray*}
  g(B) &=& k_2B\,(L-B)\\[1ex]
\end{eqnarray*}

Replace $k_2\!=\!\frac{r}{K}$ and $L=K$:

\begin{eqnarray*}
  g(B) &=& \frac{r}{K}\,B\,(K-B)       \\[1em]
  ~    &=& rB\left(\frac{K-B}{K}\right)\\[1em]
  ~    &=& rB\left(1-\frac{B}{K}\right)
\end{eqnarray*}

\newpage

\subsection{$B\msy=0.5K$}

The surplus production is maximized where the derivative of the growth function
is zero. It's easiest to find the derivative by using elementary calculus
notation,

\begin{eqnarray*}
  f(x)  &=& rx\left(1-\frac{x}{K}\right)\\[1em]
  ~     &=& rx \;-\; \frac{\;rx^2}{K}   \\[1em]
  ~     &=& rx \;-\; \frac{\,r}{K}\,x^2 \\[1em]
  f'(x) &=& r \;-\; 2\,\frac{\,r}{K}\,x \\[1em]
  ~     &=& r \;-\; \frac{2r}{K}\,x     \\[1ex]
\end{eqnarray*}

so:

\begin{eqnarray*}
  g'(B) &=& r \;-\; \frac{2r}{K}\,B\\[1ex]
\end{eqnarray*}

$B\msy$ is where the derivative equals zero,

\begin{eqnarray*}
  0 &=& r \;-\; \frac{2r}{K}\,B\msy\\[1ex]
\end{eqnarray*}

and we can isolate $B\msy$:

\begin{eqnarray*}
  \frac{2r}{K}\,B\msy &=& r            \\[1em]
  2rB\msy             &=& rK           \\[1em]
  B\msy               &=& \frac{rK}{2r}\\[1em]
  &=& \frac{r}{2r}\,K                  \\[1em]
  &=& 0.5K
\end{eqnarray*}

\newpage

\subsection{$\MSY=0.25\,rK$}

Knowing $B\msy\!=\!0.5K$ we can evaluate $\MSY\!=\!g(B\msy)$:

\begin{eqnarray*}
  g(B) &=& rB\left(1-\frac{B}{K}\right)          \\[1em]
  \MSY &=& rB\msy\!\left(1-\frac{B\msy}{K}\right)\\[1em]
  ~    &=& 0.5\,rK\!\left(1-\frac{0.5K}{K}\right)\\[1em]
  ~    &=& 0.5\,rK\,(1\!-\!0.5)                  \\[1em]
  ~    &=& 0.5\,rK\,(0.5)                        \\[1em]
  ~    &=& 0.25\,rK                              \\[1ex]
\end{eqnarray*}

\subsection{Relative growth rate}

The relative growth rate, as a fraction of current abundance, is
$\Rate(B)=g(B)/B$,

\begin{eqnarray*}
  g(B)     &=& rB\left(1-\frac{B}{K}\right)                \\[1em]
  \Rate(B) &=& \left.rB\left(1-\frac{B}{K}\right)\right/\!B\\[1em]
  ~        &=& r\left(1-\frac{B}{K}\right)                 \\[1em]
  ~        &=& r \;-\; \frac{rB}{K}                        \\[1ex]
\end{eqnarray*}

so the relative growth rate approaches $r$ when the abundance is close to zero,
and declines linearly with $B$ until zero growth occurs at abundance $K$.

\newpage

\section{Variations}

\subsection{Gompertz (1825)}

Applied in stock assessment by Fox (1970, Eq.~8):

\begin{eqnarray*}
  g(B) &=& rB\,(\log K\!-\!\log B)                                      \\[1em]
  ~    &=& rB\,\log\;\!\!\!\left(\!\frac{K}{B}\:\!\!\right)             \\[1em]
  ~    &=& rB\,\log K\!\left(\frac{\,\log K\!-\!\log B\,}{\log K}\right)\\[1em]
  ~    &=& rB\,\log K\!
  \left(\frac{\,\log K\,}{\log K}-\frac{\,\log B\,}{\log K}\right)      \\[1em]
  ~    &=& rB\,\log K\!\left(1-\frac{\,\log B\,}{\log K}\right)         \\[1em]
\end{eqnarray*}

The surplus production is maximized where the derivative of the growth function
is zero. It's easiest to find the derivative by using elementary calculus
notation,

\begin{eqnarray*}
  f(x)  &=& ax\,(b\!-\!\log x)                                  \\[1em]
  ~     &=& abx \;-\; ax\log x                                  \\[1em]
  f'(x) &=& ab \;-\; \Big[(ax)'\log x \;+\; (ax)(\log x)'\,\Big]\\[1em]
  ~     &=& ab \;-\; \Big[\,a\log x \;+\; (ax)\frac{1}{x}\,\Big]\\[1em]
  ~     &=& ab \;-\; \Big[\,a\log x \;+\; a\,\Big]              \\[1em]
  ~     &=& ab \;-\; a\log x \;-\; a                            \\[1ex]
\end{eqnarray*}

so:

\begin{eqnarray*}
  g'(B) &=& r\log K \;-\; r\log B \;-\; r
\end{eqnarray*}

\newpage

$B\msy$ is where the derivative equals zero:

\begin{eqnarray*}
  0 &=& r\log K \;-\; r\log B\msy \;-\; r\\[1ex]
\end{eqnarray*}

First isolate $\log B\msy$,

\begin{eqnarray*}
  r\log B\msy &=& r\log K \;-\; r\\[1em]
  \log B\msy &=& \log K \;-\; 1  \\[1ex]
\end{eqnarray*}

then exponentiate:

\begin{eqnarray*}
  \exp(\log B\msy) &=&       \exp(\log K-1)                  \\[1em]
  B\msy            &=&       \exp(\log K) \;\times\; \exp(-1)\\[1em]
  ~                &=&       \frac{K}{e}                     \\[1em]
  ~                &\approx& 0.368K                          \\[1ex]
\end{eqnarray*}

Knowing $B\msy\!=\!\frac{K}{e}$ we can evaluate $\MSY\!=\!g(B\msy)$:

\begin{eqnarray*}
  g(B) &=&       rB\,\log\;\!\!\!\left(\!\frac{K}{B}\:\!\!\right)       \\[1em]
  \MSY &=&       rB\msy\,\log\;\!\!\!\left(\!\frac{K}{B\msy}\!\right)   \\[1em]
  ~    &=&       r\,\frac{K}{e}\,\log\!
  \left(\!\frac{\:K\:}{\frac{K}{e}}\!\right)                            \\[1em]
  ~    &=&       \frac{rK}{e}\,\log\!\bigg(\!\frac{e}{K}\times K\!\bigg)\\[1em]
  ~    &=&       \frac{rK}{e}\,\log(e)                                  \\[1em]
  ~    &=&       \frac{rK}{e}                                           \\[1em]
  ~    &\approx& 0.368\:rK
\end{eqnarray*}

\newpage

\subsection{Pella-Tomlinson (1969)}

Pella and Tomlinson (1969, Eq.~5):

\begin{eqnarray*}
  g(B) &=& \mathcal{H}B^m \;-\; \mathcal{K}B\\[1ex]
\end{eqnarray*}

Replace $\mathcal{H}\!=\!-\frac{r}{pK^p}$, $m\!=\!p\!+\!1$, and
$\mathcal{K}\!=\!-\frac{r}{p}$:

\begin{eqnarray*}
  g(B) &=& -\frac{r}{\,pK^p\,}B^{\,p+1} \;+\; \frac{r}{p}B             \\[1em]
  ~    &=& \frac{\,rB\,}{p} \;-\; \frac{r}{pK^p}B^{\;\!p+1}            \\[1em]
  ~    &=& \frac{\,rB\,}{p} \;-\; \frac{r}{pK^p}B^p\!\times\!B         \\[1em]
  ~    &=& \frac{\,rB\,}{p} \;-\; \frac{rB}{p}\!\times\!\frac{B^p}{K^p}\\[1em]
  ~    &=& \frac{\,rB\,}{p}\left(1 \;-\; \frac{\;B^p}{\,K^p\,}\right)  \\[1em]
  ~    &=& \frac{\,rB\,}{p}\!
  \left[\,1 \;-\; \left(\frac{B}{K}\right)^{\!p}\,\right]
\end{eqnarray*}

\newpage

The surplus production is maximized where the derivative of the growth function
is zero. It's easiest to find the derivative by using elementary calculus
notation,

\begin{eqnarray*}
  f(x)  &=& \frac{ax}{c}\!
  \left[\,1-\left(\frac{x}{b}\right)^{\!c}\,\right]             \\[1em]
  ~     &=& \frac{ax}{c} \;-\;
  \frac{ax}{c}\!\left(\frac{x}{b}\right)^{\!c}                  \\[1em]
  ~     &=& \frac{ax}{c} \;-\;
  \frac{ax}{c}\!\left(\frac{x^c}{b^c}\!\right)                  \\[1em]
  ~     &=& \frac{ax}{c} \;-\;
  \frac{a}{\,c\,}\!\left(\!\frac{\,x^{\,c+1}}{b^c}\!\right)     \\[1em]
  ~     &=& \frac{ax}{c} \;-\; \frac{a}{\,c\;\!b^c\,}\,x^{\,c+1}\\[1em]
  f'(x) &=& \frac{a}{\,c\,} \;-\;
  \left(\!\frac{a}{\;c\;\!b^c\:}\!\right)\!(c\!+\!1)\,x^c       \\[1em]
  ~     &=& \frac{a}{\,c\,} \;-\;
  \left(\!\frac{ac+a}{c\;\!b^c}\!\right)x^c                     \\[1ex]
\end{eqnarray*}

so:

\begin{eqnarray*}
  g'(B) &=& \frac{r}{\,p\,} \;-\; \left(\!\frac{rp+r}{pK^p}\!\right)\!B^p\\[1ex]
\end{eqnarray*}

$B\msy$ is where the derivative equals zero:

\begin{eqnarray*}
  0 &=& \frac{r}{\,p\,} \;-\; \left(\!\frac{rp+r}{pK^p}\!\right)\!B\msy^p
\end{eqnarray*}

\newpage

First isolate $B\msy^p$,

\begin{eqnarray*}
  \left(\!\frac{rp+r}{pK^p}\!\right)\!B\msy^p &=& \frac{r}{\,p\,}\\[1em]
  B\msy^p &=& \frac{r}{\,p\,}\left(\!\frac{pK^p}{rp+r}\!\right)  \\[1em]
  ~       &=& r\left(\!\frac{K^p}{rp+r}\!\right)                 \\[1em]
  ~       &=& \frac{rK^p}{\,rp+r\,}                              \\[1em]
  ~       &=& \frac{K^p}{\;p+1\,}                                \\[1ex]
\end{eqnarray*}

then log-transform,

\begin{eqnarray*}
  \log\,(B\msy^p) &=& \log\!\left(\!\frac{K^p}{\,p\!+\!1\,}\!\right)\\[1em]
  p\log B\msy     &=& \log(K^p) \;-\; \log(p\!+\!1)\\[1em]
  \log B\msy      &=& \frac{\,p\log K \;-\; \log(p\!+\!1)\,}{p}     \\[1em]
  ~               &=& \log K \;-\; \frac{\:\log(p\!+\!1)\,}{p}      \\[1em]
  ~               &=& \log K \;+\;
  \left(\!-\frac{1}{p}\right)\log(p\!+\!1)                          \\[1em]
  ~               &=& \log K \;+\;
  \log(p\!+\!1)\!\left(\!-\frac{1}{p}\right)                        \\[1ex]
\end{eqnarray*}

and exponentiate:

\begin{eqnarray*}
  \exp(\log B\msy) &=& \exp\!\left[\log K \;+\;
    \log(p\!+\!1)\!\left(\!-\frac{1}{p}\right)\right]          \\[1em]
  B\msy            &=& \exp(\log K) \;\times\;
  \exp\!\left[\log(p\!+\!1)\!\left(\!-\frac{1}{p}\right)\right]
\end{eqnarray*}

\newpage

Noting that $x^{ab}=(x^a)^b$, we rearrange the last term:

\begin{eqnarray*}
  B\msy &=& K \;\times\;
  \exp\!\left[\log(p\!+\!1)\!\left(\!-\frac{1}{p}\right)\right]      \\[1em]
  ~     &=& K \;\times\;
  \bigg(\!\!\exp\!\Big[\!\log(p\!+\!1)\Big]\bigg)^{-\frac{1}{p}}     \\[1em]
  ~     &=& K \;\times\; (p\!+\!1)^{-\frac{1}{p}}\\[1em]
  ~     &=& K\!\left(\!\frac{1}{\,p\!+\!1\,}\!\right)^{\!\frac{1}{p}}\\[1ex]
\end{eqnarray*}

When $p\!=\!1$ we get Schaefer:

\begin{eqnarray*}
  B\msy &=& K\!\left(\!\frac{1}{\,1\!+\!1\,}\!\right)^{\!\frac{1}{1}}\\[1em]
  ~     &=& K\!\left(\!\frac{1}{\,2\,}\!\right)                      \\[1em]
  ~     &=& 0.5K
\end{eqnarray*}

\newpage

When $p\!\to\!0$ we get Gompertz,

\begin{eqnarray*}
  B\msy &=& \lim_{p\to 0}\,K\!
  \left(\!\frac{1}{\,p\!+\!1\,}\!\right)^{\!\frac{1}{p}}              \\[1em]
  ~     &=& K \;\times\; \lim_{p\to 0}
  \left(\!\frac{1}{\,1\!+\!p\,}\!\right)^{\!\frac{1}{p}}              \\[1em]
  ~     &=& K \;\times\; \lim_{p\to 0}
  \left[\Big(1\!+\!p\Big)^{\!-1}\right]^{\frac{1}{p}}                 \\[1em]
  ~     &=& K \;\times\;
  \lim_{p\to 0}\Big(1\!+\!p\Big)^{\!-\frac{1}{p}}\\[1em]
  ~     &=& K \;\times\; \exp\!\left(\lim_{p\to 0}\,
    \log\!\left[\Big(1\!+\!p\Big)^{\!-\frac{1}{p}}\right]\right)      \\[1em]
  ~     &=& K \;\times\; \exp\!
  \left[\,\lim_{p\to 0}\,-\,\frac{1}{\,p\,}\log(1\!+\!p)\right]       \\[1em]
  ~     &=& K \;\times\; \exp\!
  \left[\,-\lim_{p\to 0}\frac{\,\log(p\!+\!1)\,}{p}\right]            \\[1ex]
\end{eqnarray*}

noting that $\displaystyle\lim_{x\to
  c}\,\frac{f(x)}{g(x)}\,=\,\displaystyle\lim_{x\to c}\,\frac{f'(x)}{g'(x)}\:$:

\begin{eqnarray*}
  B\msy &=&       K \;\times\;
  \exp\!\left[\,-\lim_{p\to 0}\frac{\,\frac{1}{1+p}\,}{1}\right]\\[1em]
  ~     &=&       K \;\times\;
  \exp\!\left[\,-\lim_{p\to 0}\,\frac{1}{\,1\!+\!p\;}\right]    \\[1em]
  ~     &=&       K \;\times\;
  \exp\!\left[\,-\,\frac{1}{\,1\!+\!0\;}\right]                 \\[1em]
  ~     &=&       K \;\times\; \exp(-1)                         \\[1em]
  ~     &=&       \frac{K}{e}                                   \\[1em]
  ~     &\approx& 0.368K
\end{eqnarray*}

\newpage

Knowing $B\msy=K\!\left(\!\frac{1}{\,p+1\,}\!\right)^{\!\frac{1}{p}}$ we can
evaluate $\MSY\!=\!g(B\msy)$:

\begin{eqnarray*}
  g(B) &=& \frac{\,rB\,}{p}\!
  \left[\,1 \;-\; \left(\frac{B}{K}\right)^{\!p}\,\right]                \\[1em]
  \MSY &=& \frac{\,rB\msy\,}{p}\!
  \left[\,1 \;-\; \left(\frac{B\msy}{K}\right)^{\!p}\,\right]            \\[1em]
  ~    &=&
  \frac{\,r\,K\!\left(\!\frac{1}{\,p+1\,}\!\right)^{\!\frac{1}{p}}}{p}
  \left[\,1 \;-\; \left(\!\frac{\,K\!
        \left(\!\frac{1}{\,p+1\,}\!\right)^{\!\frac{1}{p}}\,}{K}\!\right)
    ^{\!p}\;\right]                                                      \\[1em]
  ~    &=& \frac{\,rK\,}{p}\!
  \left(\!\frac{1}{\,p\!+\!1\,}\!\right)^{\!\frac{1}{p}}\,
  \left(1 \;-\; \left[\left(\!\frac{1}{\,p+1\,}\!\right)
      ^{\!\frac{1}{p}}\right]^{\!p}\;\right)                             \\[1em]
  ~    &=& \frac{\,rK\,}{p}\!
  \left(\!\frac{1}{\,p\!+\!1\,}\!\right)^{\!\frac{1}{p}}\,
  \left(1 \;-\; \!\frac{1}{\,p+1\,}\!\right)                             \\[1em]
  ~    &=& \frac{\,rK\,}{p}\!
  \left(\!\frac{1}{\,p\!+\!1\,}\!\right)^{\!\frac{1}{p}}\,
  \left(\frac{\,p+1\,}{p+1} \;-\; \!\frac{1}{\,p+1\,}\!\right)           \\[1em]
  ~    &=& \frac{\,rK\,}{p}\!
  \left(\!\frac{1}{\,p\!+\!1\,}\!\right)^{\!\frac{1}{p}}\,
  \left(\!\frac{p}{\,p+1\,}\!\right)                                     \\[1em]
  ~    &=& \frac{\,rK\,}{p}\!\left(\!\frac{p}{\,p+1\,}\!\right)\!
  \left(\!\frac{1}{\,p\!+\!1\,}\!\right)^{\!\frac{1}{p}}                 \\[1em]
  ~    &=& rK\!\left(\!\frac{1}{\,p\!+\!1\,}\!\right)\!
  \left(\!\frac{1}{\,p+1\,}\!\right)^{\!\frac{1}{p}}                     \\[1em]
  ~    &=& rK\!\left(\!\frac{1}{\,p+1\,}\!\right)^{\!1+\frac{1}{p}}      \\[1ex]
\end{eqnarray*}

When $p\!=\!1$ we get Schaefer:

\begin{eqnarray*}
  \MSY &=& rK\left(\!\frac{1}{\,1+1\,}\!\right)^{\!1+\frac{1}{1}}\\[1em]
  ~    &=& rK\left(\frac{1}{2}\right)^{\!2}                      \\[1em]
  ~    &=& 0.25\,rK
\end{eqnarray*}

\newpage

When $p\!\to\!0$ we get Gompertz,

\begin{eqnarray*}
  \MSY &=& \lim_{p\to 0}\,
  rK\!\left(\!\frac{1}{\,p\!+\!1\,}\!\right)^{\!1+\frac{1}{p}}        \\[1em]
  ~    &=& rK \;\times\; \lim_{p\to 0}
  \left(\!\frac{1}{\,1\!+\!p\,}\!\right)^{\!1+\frac{1}{p}}            \\[1em]
  ~    &=& rK \;\times\; \exp\!\left(\lim_{p\to 0}\,
    \log\left[\left(\!\frac{1}{\,1\!+\!p\,}\!\right)^{\!1+\frac{1}{p}}
    \right]\right)                                                    \\[1em]
  ~    &=& rK \;\times\; \exp\!\left[\,\lim_{p\to 0}
    \left(\!1\!+\!\frac{1}{p}\right)
    \log\!\left(\!\frac{1}{\,1\!+\!p\,}\!\right)\right]               \\[1em]
  ~    &=& rK \;\times\; \exp\!\left[\,-\lim_{p\to 0}
    \left(\!1\!+\!\frac{1}{p}\right)\log(1\!+\!p)\right]              \\[1em]
  ~    &=& rK \;\times\; \exp\!\left[\,-\lim_{p\to 0}\;
    \log(1\!+\!p) \;+\; \frac{\,\log(1\!+\!p)\,}{p}\right]            \\[1em]
  ~    &=& rK \;\times\; \exp\!\left[\,-\lim_{p\to 0}
    \frac{\,\log(1\!+\!p)\,}{p}\right]                                \\[1ex]
\end{eqnarray*}

noting that $\displaystyle\lim_{x\to
  c}\,\frac{f(x)}{g(x)}\,=\,\displaystyle\lim_{x\to c}\,\frac{f'(x)}{g'(x)}\:$:

\begin{eqnarray*}
  \MSY &=&      rK \;\times\;
  \exp\!\left[\,-\lim_{p\to 0} \frac{\,\log(1\!+\!p)\,}{p}\right]\\[1em]
  ~    &=&        rK \;\times\;
  \exp\!\left[\,-\lim_{p\to 0}\frac{\,\frac{1}{1+p}\,}{1}\right]\\[1em]
  ~    &=&       rK \;\times\;
  \exp\!\left[\,-\lim_{p\to 0}\,\frac{1}{\,1\!+\!p\;}\right]    \\[1em]
  ~    &=&       rK \;\times\;
  \exp\!\left[\,-\,\frac{1}{\,1\!+\!0\;}\right]                 \\[1em]
  ~    &=&       rK \;\times\; \exp(-1)                         \\[1em]
  ~    &=&       \frac{rK}{e}                                   \\[1em]
  ~    &\approx& 0.368\,rK
\end{eqnarray*}

\newpage

The Pella-Tomlinson shape parameter $p$ does not only shift $B\msy$ away from
$0.5K$, but also changes the maximum height of the production curve. This is
unfortunate, as the modeller might be interested in exploring different
shapes without altering the maximum productivity.\\[1em]

Since we know the maximum surplus production for a given $p$,

\begin{eqnarray*}
  \MSY &=& rK\!\left(\!\frac{1}{\,p+1\,}\!\right)^{\!1+\frac{1}{p}}\\[1ex]
\end{eqnarray*}

the production can be scaled to any given range.\\[1em]

Starting from the original Pella-Tomlinson,

\begin{eqnarray*}
  g(B) &=& \frac{rB}{p}\!\left[\,1 \;-\; \left(\frac{B}{K}\right)^{\!p}\,\right]\\[1ex]
\end{eqnarray*}

we first divide by MSY to scale the production between 0 and 1,

\newpage

\begin{eqnarray*}
  g_1(B) &=& \frac{\quad\frac{rB}{p}\!
    \left[\,1-\left(\frac{B}{K}\right)^p\,\right]\quad}
  {rK\!\left(\!\frac{1}{\,p+1\,}\!\right)^{\!1+\frac{1}{p}}}            \\[1em]
  ~    &=& \frac{\quad\frac{B}{p}\!
    \left[\,1-\left(\frac{B}{K}\right)^p\,\right]\quad}
  {K\!\left(\!\frac{1}{\,p+1\,}\!\right)^{\!1+\frac{1}{p}}}             \\[1em]
  ~    &=& \frac{\quad B\!
    \left[\,1-\left(\frac{B}{K}\right)^p\,\right]\quad}
  {pK\!\left(\!\frac{1}{\,p+1\,}\!\right)^{\!1+\frac{1}{p}}}            \\[1em]
  ~    &=& \frac{\quad B\!
    \left[\,1-\left(\frac{B}{K}\right)^p\,\right]\quad}
  {pK\!\left[\!\frac{1}{\,(p+1)^{\!1+\frac{1}{p}}\,}\!\right]}          \\[1em]
  ~    &=& \frac{\quad B\!
    \left[\,1-\left(\frac{B}{K}\right)^p\,\right]\quad}
  {\frac{pK}{\,(p+1)^{\!1+\frac{1}{p}}\,}}                              \\[1em]
  ~    &=& \frac{\;(p\!+\!1)^{1+\frac{1}{p}}\,}{pK} \quad\times\quad
  B\!\left[\,1-\left(\frac{B}{K}\right)^p\,\right]                      \\[1em]
  ~    &=& \frac{B}{\,pK\,}\,(p\!+\!1)^{1+\frac{1}{p}}
  \!\left[\,1-\left(\frac{B}{K}\right)^p\,\right]                       \\[1ex]
\end{eqnarray*}

and then multiply by $0.25\,rK$ to use that as maximum production at any given
$p$:

\begin{eqnarray*}
  g^\star(B) &=& 0.25\,rK \;\times\;
  \frac{B}{\,pK\,}\,(p\!+\!1)^{1+\frac{1}{p}}\!
  \left[\,1-\left(\frac{B}{K}\right)^p\,\right]                \\[1em]
  ~ &=& \frac{\,0.25\,rB\,}{\,p\,}\,(p\!+\!1)^{1+\frac{1}{p}}\!
  \left[\,1-\left(\frac{B}{K}\right)^p\,\right]                \\[1ex]
\end{eqnarray*}

\newpage

\subsection{Theta-logistic (1973)}

Gilpin and Ayala (1973, Eq.~3):

\begin{eqnarray*}
  g(B) &=& rB\!\left[\,1\!-\!\left(\frac{B}{K}\right)^{\!\theta}\:\right]\\[1ex]
\end{eqnarray*}

where $\theta$ gives the asymmetry of the growth.\\[1em]

Different from Pella-Tomlinson, which has $\frac{r}{p}$ as the first term. The
theta-logistic model has the nice property that $r$ is the initial growth rate,
independent of $\theta$. The Pella-Tomlinson model has the nice property that it
becomes the Gompertz model as $p\to 0$.\\[1em]

The theta-logistic model has a problem with low $\theta$ values. Maximum
production tends towards zero with decreasing $\theta$, so the model will need
very large $K$ when $\theta$ is small.

\newpage

\subsection{Fletcher (1978)}

From Prager (2002, Eqs~2--4):

\begin{eqnarray*}
  g(B) &=& \gamma m\frac{B}{K} \;-\;
  \gamma m\left(\frac{B}{K}\right)^{\!n}       \\[1em]
  ~    &=& \gamma m\left[\,\frac{B}{K} \;-\;
    \left(\frac{B}{K}\right)^{\!\!n}\,\right]  \\[1em]
  ~    &=& \gamma m\frac{B}{K}\left[\,1 \;-\;
    \left(\frac{B}{K}\right)^{\!\!n-1}\,\right]\\[1ex]
\end{eqnarray*}

where

\begin{eqnarray*}
  \gamma &=& \frac{n^{n/(n-1)}}{n-1}\\[1ex]
\end{eqnarray*}

and $m\!=\!\frac{1}{4}\,rK$, i.e.\ MSY:

\begin{eqnarray*}
  g(B) &=& \gamma m\left[\,\frac{B}{K} \;-\;
    \left(\frac{B}{K}\right)^{\!\!n}\,\right]              \\[1em]
  ~    &=& \gamma\,\frac{rK}{4}\left[\,\frac{B}{K} \;-\;
    \left(\frac{B}{K}\right)^{\!\!n}\,\right]              \\[1em]
  ~    &=& \gamma\,\frac{rK}{4}\,\frac{B}{K} \;-\;
  \gamma\,\frac{rK}{4}\!\left(\frac{B}{K}\right)^{\!\!n}   \\[1em]
  ~    &=& \gamma\,\frac{r}{4}\,B \;-\;
  \gamma\,\frac{rK}{4}\!\left(\frac{B}{K}\right)^{\!\!n}   \\[1em]
  ~    &=& \gamma\,\frac{1}{4}\,rB \;-\;
  \gamma\,\frac{1}{4}\,rK\!\left(\frac{B}{K}\right)^{\!\!n}\\[1em]
  ~    &=& \frac{\gamma}{4}\,rB \;-\;
  \frac{\gamma}{4}\,rK\!\left(\frac{B}{K}\right)^{\!\!n}   \\[1em]
  ~    &=& \frac{\gamma}{4}\,r\left[B \;-\;
    rK\!\left(\frac{B}{K}\right)^{\!\!n}\right]
\end{eqnarray*}

\newpage

The Schaefer model corresponds to $n\!=\!2$ where $\gamma\!=\!4$,

\begin{eqnarray*}
  \gamma &=& \frac{n^{n/(n-1)}}{n-1}\\[1em]
  ~      &=& \frac{2^{2/(2-1)}}{2-1}\\[1em]
  ~      &=& \frac{2^{2/(1)}}{1}    \\[1em]
  ~      &=& 2^{2/(1)}              \\[1em]
  ~      &=& 2^2                    \\[1em]
  ~      &=& 4                      \\[1ex]
\end{eqnarray*}

as verified here:

\begin{eqnarray*}
  g(B) &=& \frac{\gamma}{4}\,rB \;-\;
  \frac{\gamma}{4}\,rK\!\left(\frac{B}{K}\right)^{\!\!n}\\[1em]
  ~    &=& \frac{4}{4}\,rB \;-\;
  \frac{4}{4}\,rK\!\left(\frac{B}{K}\right)^{\!2}       \\[1em]
  ~    &=& rB \;-\; rK\!\left(\frac{B}{K}\right)^{\!2}  \\[1em]
  ~    &=& rB \;-\; rK\,\frac{B^2}{K^2}                 \\[1em]
  ~    &=& rB \;-\; r\,\frac{\;B^2}{K}                  \\[1em]
  ~    &=& rB \;-\; rB\,\frac{B}{K}                     \\[1em]
  ~    &=& rB\left(1 \;-\; \frac{B}{K}\right)
\end{eqnarray*}

\newpage

Prager (2002) describes the shape of the production curve with the unitless
ratio $\phi\!=\!\frac{B\msy}{K}$, which has a more intuitive meaning than the
$n$ exponent. The relationship is:

\begin{eqnarray*}
  \phi &=& \left(\frac{1}{n}\right)^{\!1/(n-1)}\\[1ex]
\end{eqnarray*}

Insert $n\!=\!2$ and note how the Schaefer model corresponds to $\phi=0.5$, as
expected:

\begin{eqnarray*}
  \phi &=& \left(\frac{1}{n}\right)^{\!1/(n-1)}\\[1em]
  ~    &=& \left(\frac{1}{2}\right)^{\!1/(2-1)}\\[1em]
  ~    &=& \left(\frac{1}{2}\right)^{\!1/(1)}\\[1em]
  ~    &=& \frac{1}{2}
\end{eqnarray*}

\subsection{Polacheck et al.\ (1993)}

Polacheck et al. (1993, Eq.~1):

\begin{eqnarray*}
  g(B) &=& \frac{r}{p}\,B\!
  \left[\,1\!-\!\left(\frac{B}{K}\right)^{\!p}\,\right]\\[1ex]
\end{eqnarray*}

where $p$ controls the asymmetry of the sustainable yield versus stock biomass
relationship.\\[1em]

The authors note that the $\frac{r}{p}$ term is often omitted when the formula
is presented.

\newpage

\section{References}

\small\sloppy\frenchspacing\setlength{\hyphenpenalty}{1000}
\begin{description}
  \item Fox, W.W., Jr. 1970. An exponential surplus-yield model for optimizing
  exploited fish populations. Trans. Am. Fish. Soc. 99:80--88.

  \item Garrod, D.J. 1969. Empirical assessments of catch effort relationships
  in the North Atlantic cod stock. ICNAF Res. Bull. 6:26--34.

  \item Gilpin, M.E. and F.J. Ayala. 1973. Global models of growth and
  competition. Proc. Natl. Acad. Sci. USA 70:3590--3593.

  \item Gompertz, B. 1825. On the nature of the function expressive of the law
  of human mortality, and on a new mode of determining the value of life
  contingencies. Phil. Trans. R. Soc. Lond. 115:513--583.

  \item Kingsland, S. 1982. The refractory model: The logistic curve and the
  history of population ecology. Q. Rev. Biol. 57:29--52.

  \item Laloe, F. 1995. Should surplus production models be fishery description
  tools rather than biological models? Aquat. Living Resour. 8:1--16.

  \item Pella, J.J. and P.K. Tomlinson. 1969. A generalized stock production
  model. IATTC Bull. 13:421--496.

  \item Polacheck, T., R. Hilborn, and A.E. Punt. 1993. Fitting surplus
  production models: Comparing methods and measuring uncertainty. Can. J. Fish.
  Aquat. Sci. 50:2597--2607.

  \item Prager, M.H. 2002. Comparison of logistic and generalized
  surplus-production models applied to swordfish, Xiphias gladius, in the north
  Atlantic Ocean. Fish. Res. 58:41--57.

  \item Schaefer, M.B. 1954. Some aspects of the dynamics of populations
  important to the management of the commercial marine fisheries. IATTC Bull.
  1:27--56.
\end{description}

\subsection{Look up later}
\small\sloppy\frenchspacing\setlength{\hyphenpenalty}{1000}
\begin{description}
  \item Fletcher, R.I. 1978. On the restructuring of the Pella-Tomlinson system.
  Fish. Bull. 76:515--521.

  \item Fox, W.W., Jr. 1970. An exponential surplus-yield model for optimizing
  exploited fish populations. Trans. Am. Fish. Soc. 99:80--88.

  \item Garrod, D.J. 1969. Empirical assessments of catch effort relationships
  in the North Atlantic cod stock. ICNAF Res. Bull. 6:26--34.

  \item Hilborn, R. and C.J. Walters. 1992. Quantitative fisheries stock
  assessment: Choice, dynamics and uncertainty. New York: Chapman and Hall.

  \item Quinn, T.J., II and R.B. Deriso. 1999. Quantitative fish dynamics. New
  York: Oxford University Press.
\end{description}

\end{document}
